\begin{abstract}
\textit{\textcolor{darkgray}{Boat owners have to take special care of cleaning their boats, as the fouling continuously accumulating on the hulls greatly increases hydrodynamic drag. Without taking this special care, a higher fuel consumption or lower velocity of the vessel is observed. This paper investigates ways for an autonomous robot to navigate large ship-hulls for cleaning purposes. This investigation is done by simulation in Webots, wherein the robot H.A.C.K. is used to map the surface of a number of different simple geometric shapes, with the goal of minimizing the coverage over time. A number of different navigation methods are tested, such as a Random Walk Method, wherein the robot changes its heading vector to a random angle, an Augmented Random Walk; wherein the robot, in addition to the previous method, also changes heading randomly during its planned path, even without hitting an edge; Beacon-Based Algorithm (BBA), wherein the robot is localising itself using GPS readings sent from beacons, that additionally help in identifying the boundaries and limit the robot’s workspace; Slicer algorithm, wherein the robot has the global map derived from the ship-model and the path is generated in a similar way to how a resin 3D printer generates its tool-path. The test results is analysed using Matlab to calculate the coverage over time and an overlapping coverage, for later optimisation purposes. To obtain the best results, water physics are applied to the simulation, and the tests are performed 18 times over a 3 hour period, with in the simulation, to get a representative result.}}
\end{abstract}